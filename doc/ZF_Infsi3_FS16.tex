\documentclass[10pt,a4paper]{article}
\usepackage[utf8]{inputenc}
\usepackage{fancyhdr}
\usepackage{ngerman}
\usepackage{floatflt}
\usepackage{float}
\usepackage{graphicx}
\usepackage{tabularx}
\usepackage[ampersand]{easylist} % Aufzählung mit &
\usepackage{amsmath}
\usepackage{amsfonts}
\usepackage{amssymb}
\usepackage{array}
\usepackage{framed}
\usepackage[hidelinks]{hyperref}
\usepackage[left=3.5cm,right=2cm,top=2cm,bottom=2cm,includeheadfoot]{geometry}
\reversemarginpar
\pagestyle{fancy} %eigener Seitenstil
\setlength{\parindent}{0pt}

%Formatierung der Tabellen
\usepackage[table]{xcolor}
\renewcommand{\arraystretch}{1.3} % Zeilenabstand für Tabellen
\setlength\arrayrulewidth{0.7pt}

\usepackage{color}

%Farben für die Syntaxhervorhebung
\definecolor{pblue}{rgb}{0.13,0.13,1}
\definecolor{pgreen}{rgb}{0,0.5,0}
\definecolor{pred}{rgb}{0.9,0,0}
\definecolor{pgrey}{rgb}{0.46,0.45,0.48}

% Pakete für die Code-Formatierung von Programmcode
\usepackage{todonotes}
\usepackage{listings}

\renewcommand{\ttdefault}{pcr}
\lstdefinelanguage{pseudo}{
  morekeywords={if, then, else, for, in, remove, from, case, do, forever, to, return, False, True, Algorithm, Input, Output},
  literate={<=} {$\le$}{2} {!=} {$\neq$}{2} {=} {$\leftarrow$}{2} {==} {=}{2} {&&} {$\cap$}{2} {||} {$\cup$}{2},
  commentstyle=\textit,
  keywordstyle=\bfseries,
  sensitive=true,%
  morecomment=[s]{\{}{\}},%
  morestring=[b]',%
}
\lstset{language=Java,
  numbers=left,
  frame=none,
  showspaces=false,
  showtabs=false,
  breaklines=true,
  showstringspaces=false,
  breakatwhitespace=true,
  commentstyle=\color{pgreen},
  keywordstyle=\color{pblue},
  stringstyle=\color{pred},
  basicstyle=\ttfamily,
  moredelim=[il][\textcolor{pgrey}]{!!},
  moredelim=[is][\textcolor{pgrey}]{\%\%}{\%\%}
}

\lstset{literate=
  {á}{{\'a}}1 {é}{{\'e}}1 {í}{{\'i}}1 {ó}{{\'o}}1 {ú}{{\'u}}1
  {Á}{{\'A}}1 {É}{{\'E}}1 {Í}{{\'I}}1 {Ó}{{\'O}}1 {Ú}{{\'U}}1
  {à}{{\`a}}1 {è}{{\`e}}1 {ì}{{\`i}}1 {ò}{{\`o}}1 {ù}{{\`u}}1
  {À}{{\`A}}1 {È}{{\'E}}1 {Ì}{{\`I}}1 {Ò}{{\`O}}1 {Ù}{{\`U}}1
  {ä}{{\"a}}1 {ë}{{\"e}}1 {ï}{{\"i}}1 {ö}{{\"o}}1 {ü}{{\"u}}1
  {Ä}{{\"A}}1 {Ë}{{\"E}}1 {Ï}{{\"I}}1 {Ö}{{\"O}}1 {Ü}{{\"U}}1
  {â}{{\^a}}1 {ê}{{\^e}}1 {î}{{\^i}}1 {ô}{{\^o}}1 {û}{{\^u}}1
  {Â}{{\^A}}1 {Ê}{{\^E}}1 {Î}{{\^I}}1 {Ô}{{\^O}}1 {Û}{{\^U}}1
  {œ}{{\oe}}1 {Œ}{{\OE}}1 {æ}{{\ae}}1 {Æ}{{\AE}}1 {ß}{{\ss}}1
  {ç}{{\c c}}1 {Ç}{{\c C}}1 {ø}{{\o}}1 {å}{{\r a}}1 {Å}{{\r A}}1
  {€}{{\EUR}}1 {£}{{\pounds}}1
}

\makeatletter
\let\ps@plain\ps@fancy 
\makeatother

\title{InfSi3 Zusammenfassung}

\author{Roman Ehrbar}

\date{\today}

\begin{document}
\maketitle
\tableofcontents
\listoftodos
\newpage 

\section{Einleitung}

\todo[inline]{Gesetzliche Anforderungen}
\todo[inline]{Bedrohungen}

\section{Secure Software / Software Security}

Durch das Aufkommen des Internets nahm auch die vom \textit{National Institute of Standards and Technology} (NIST) erfassten Vulnerabilities mit hohem Ausmass zu.

\subsection{Most Dangerous Software Errors}
Die 25 häufigsten Schwachstellen können in drei Kategorien eingeteilt werden:
\begin{itemize}
	\item Ausgelöst durch unsichere Wege, in denen die Daten gesendet und empfangen werden. Dazu zählen \textit{SQL Injection}, \textit{OS Command Injection}, \textit{XSS}, \textit{CSRF} und \textit{Open Redirect}.
	\item Ausgelöst durch die unsichere Handhabung von Systemressourcen. Dazu gehören \textit{Classic Buffer Overflow}, \textit{Path Traversal}, \textit{Uncontrolled Format String}, \textit{Potentially Dangerous Function} und \textit{Integer Overflow}.
	\item Ausgelöst durch falsche Verwendung, Missbrauch oder Ignorierung von defensiven Sicherheitsmechanismen. Dazu zählen \textit{Missing Authentication}, \textit{Missing/Incorrect Authorization}, \textit{Hard-Coded Credentials}, \textit{Missing Encryption}, \textit{Reliance on Untrusted Inputs}, \textit{Incorrect Permission Assignment}, \textit{Use of broken or risky Cryptographic algorithm} und \textit{One-Way Hash without Salt}.
\end{itemize}

\subsection{Trinity of Trouble}
Folgende drei Aspekte sind mitunter verantwortlich für diese Schwierigkeiten.
\begin{description}
	\item[Connectivity] Immer mehr Systeme sind über das Internet miteinander verbunden und eröffnen somit neue \textit{Attack Vectors}. \textit{Service Oriented Architecture (SOA)} führt alte Systeme, welche nicht für die Vernetzung vorgesehen wurden, zusammen und veröffentlicht diese.
	\item[Extensibility] Systeme sind erweiterbar, wodurch ein teil der Kontrolle abgegeben wird. Über schlecht gewartete Erweiterungen können so neue Schwachstellen entstehen.
	\item[Complexity] Moderne Software wird immer komplexer. Mit dem Umfang nimmt auch die Fehlerrate quadratisch zu.
\end{description}

\subsection{Bugs + Flaws = Defects}
\begin{description}
	\item[Security Bug] Implementation-level Schwachstelle
	\item[Security Flaw] Design-level Schwachstelle (können selten automatisiert erkannt werden)
	\item[Security Defect] Ruhender defekt in der Software, welcher durch ein Bug oder Flaw ausgelöst wird.
\end{description}

\subsection{Software Artefakte}
Es gibt ein gemeinsames Set von Artefakten, welche unabhängig vom eigentlichen Entwicklungsprozess (Scrum, RUP, XP, \ldots) sind. Diese sind:
\begin{easylist}[itemize]
	& Anforderungen und Use Cases
	& Architektur und Design
	& Testpläne
	& Code
	& Tests und Testresultate
	& Rückmeldung von Kunden
\end{easylist}

\subsection{Drei Säulen der Software Security}

Zu den zentralen drei Säulen gehören \textbf{Risiko Management}, \textbf{Best Practices} und \textbf{Fachwissen}.

\subsubsection{Risiko Management}
Man identifiziert die betroffenen Personen, die technischen Risiken, auch die für das Unternehmen, und priorisiert sie anhand der gewonnenen Informationen. Danach kann eine Strategie zur Minderung entwickelt werden. Nach der Anwendung sollten die Anpassungen auf ihre Wirkung hin überprüft werden.

\begin{figure}[H]
	\includegraphics[width=0.6\textwidth]{./img/risk-evaluation}
	\caption{Risiko Evaluations-Matrix}
\end{figure}

\subsubsection{Best Practices}

Es folgen einige Best Practices in der Reihenfolge ihrer Effektivität. Sie können den Kategorien \textbf{K}onstruktiv (white hat) und \textbf{D}estruktiv (black hat) zugeordnet werden.
\begin{easylist}
	& Code Review \textbf{K}
	& Architectural risk analysis (historical knowledge) \textbf{K D}
	& Penetration testing \textbf{D}
	& Risk-based security tests \textbf{D K}
	& Abuse cases \textbf{D K}
	& Security requirements \textbf{K}
	& Security operation \textbf{K}
\end{easylist}

\subsubsection{Fachwissen}
Zum Fachwissen gibt es mehrere Perspektiven. Das Wissen über die Prinzipien, Rahmenbedingungen und Regeln gehören zum Vorgeschriebenen Fachwissen. Dazu gesellt sich die Diagnostischen Fähigkeiten mit dem Wissen über Angriffe, Schwachstellen und Angriffsmuster. Zuletzt benötigt man auch ein Wissen über die Vergangenheit.

\begin{figure}[H]
	\includegraphics[width=\textwidth]{./img/sdl-best-practice}
	\caption{Best Practices und Fachwissen angewendet auf die Software Artefakte}
\end{figure}

\subsection{Code Analyse}

Für die Analyse von Sourcecode bieten sich mehrere Lösungen an, welche den Quellcode und auch das laufende Programm analysieren und auswerten können. Die ersten Generationen von solchen Analyseprogrammen wurden auch als \textit{Intelligentes Grep} bezeichnet und lieferten viele false positives. In späteren, oft kommerziellen Tools, wurden diese, durch Parsen des Source-Codes, versucht zu minimieren.

Das Wissen aus der Vergangenheit ist in die Regelsätze dieser Software eingeflossen. Jedoch können Probleme in der Softwarearchitektur nicht oder nur selten erkannt werden. Ein manuelles Review ist somit weiterhin nötig. Auch können Abhängigkeiten und falsche Benutzung externer Komponenten nicht korrekt überprüft und erkannt werden.

\todo[inline]{Code-Analyse}
\todo[inline]{SDL}

\section{Application Security Basics}


\todo[inline]{HTTP}
\todo[inline]{Redirect after successful login}
\todo[inline]{Cookies \& Javascript}
\todo[inline]{Session Handling}
\todo[inline]{Same Origin Policy}
\todo[inline]{CORS}

\section{OWASP Top 10}
Die OWASP Top 10 sind die zehn wohl wichtigsten Schwachstellen von Anwendungen, welche aus dem \textit{Open Web Application Security Project} hervor ging.\\

Die aktuellste Version ist 2013 in \href{http://owasptop10.googlecode.com/files/OWASP\%20Top\%2010\%20-\%202013.pdf}{Englisch} erschienen und in mehrere Sprachen übersetzt, unter anderem auch \href{https://www.owasp.org/images/4/42/OWASP_Top_10_2013_DE_Version_1_0.pdf}{Deutsch}. Die hier aufgeführten Kurzbeschreibungen stammen ebenfalls aus der deutschen Version.

\subsection{Injection}
Injection-Schwachstellen (SQL-, OS- oder LDAP-Injection) treten auf wenn nicht vertrauenswürdige Daten als Teil eines Kommandos oder einer Abfrage von einem Interpreter verarbeitet werden. Ein Angreifer kann Eingabedaten dann so manipulieren, dass er nicht vorgesehene Kommandos ausführen oder unautorisiert auf Daten zugreifen kann.

\subsection{Broken Authentication and Session management}
Anwendungsfunktionen, die die Authentifizierung und das Session-Management umsetzen, werden oft nicht korrekt implementiert. Dies erlaubt es Angreifern Passwörter oder Session-Token zu kompromittieren oder die Schwachstellen so auszunutzen, dass sie die Identität anderer Benutzer annehmen können.

\subsection{Cross-Site-Scripting - XSS}
XSS-Schwachstellen treten auf, wenn eine Anwendung nicht vertrauenswürdige Daten entgegennimmt und ohne entsprechende Validierung oder Umkodierung an einen Webbrowser sendet. XSS erlaubt es einem Angreifer Scriptcode im Browser eines Opfers auszuführen und somit Benutzersitzungen zu übernehmen, Seiteninhalte zu verändern oder den Benutzer auf bösartige Seiten umzuleiten.

\subsection{Insecure Direct Object References}
Unsichere direkte Objektreferenzen treten auf, wenn Entwickler Referenzen zu internen Implementierungsobjekten, wie Dateien, Ordner oder Datenbankschlüssel von aussen zugänglich machen. Ohne Zugriffskontrolle oder anderen Schutz können Angreifer diese Referenzen manipulieren um unautorisiert Zugriff auf Daten zu erlangen.

\subsection{Security Misconfiguration}
Sicherheit erfordert die Festlegung und Umsetzung einer sicheren Konfiguration für Anwendungen, Frameworks, Applikations-, Web- und Datenbankserver sowie deren Plattformen. Sicherheitseinstellungen müssen definiert, umgesetzt und gewartet werden, die Voreinstellungen sind oft unsicher. Des Weiteren umfasst dies auch die regelmässige Aktualisierung aller Software.

\subsection{Sensitive Data Exposure}
Viele Anwendungen schützen sensible Daten, wie Kreditkartendaten oder Zugangsinformationen nicht ausreichend. Angreifer können solche nicht angemessen geschützten Daten auslesen oder modifizieren und mit ihnen weitere Straftaten, wie beispielsweise Kreditkartenbetrug, oder Identitätsdiebstahl begehen. Vertrauliche Daten benötigen zusätzlichen Schutz, wie z.B. Verschlüsselung während der Speicherung oder Übertragung sowie besondere Vorkehrungen beim Datenaustausch mit dem Browser.

\subsection{Missing Function Level Access Control}
Die meisten betroffenen Anwendungen realisieren Zugriffsberechtigungen nur durch das Anzeigen oder Ausblenden von Funktionen in der Benutzeroberfläche. Allerdings muss auch beim direkten Zugriff auf eine geschützte Funktion eine Prüfung der Zugriffsberechtigung auf dem Server stattfinden, ansonsten können Angreifer durch gezieltes Manipulieren von Anfragen ohne Autorisierung trotzdem auf diese zugreifen.

\subsection{Cross-Site Request Forgery - CSRF}
Ein CSRF-Angriff bringt den Browser eines angemeldeten Benutzers dazu, einen manipulierten HTTP-Request an die verwundbare Anwendung zu senden. Session Cookies und andere Authentifizierungsinformationen werden dabei automatisch vom Browser mitgesendet. Dies erlaubt es dem Angreifer Aktionen innerhalb der betroffen Anwendungen im Namen und Kontext des angegriffen Benutzers auszuführen.

\subsection{Using Components with Known Vulnerabilities}
Komponenten wie z.B. Bibliotheken, Frameworks oder andere Softwaremodule werden meistens mit vollen Berechtigungen ausgeführt. Wenn eine verwundbare Komponente ausgenutzt wird, kann ein solcher Angriff zu schwerwiegendem Datenverlust oder bis zu einer Serverübernahme führen. Applikationen, die Komponenten mit bekannten Schwachstellen einsetzen, können Schutzmassnahmen unterlaufen und so zahlreiche Angriffe und Auswirkungen ermöglichen.

\subsection{Unvalidated Redirects and Forwards}
Viele Anwendungen leiten Benutzer auf andere Seiten oder Anwendungen um oder weiter. Dabei werden für die Bestimmung des Ziels oft nicht vertrauenswürdige Daten verwendet. Ohne eine entsprechende Prüfung können Angreifer ihre Opfer auf Phishing-Seiten oder Seiten mit Schadcode um- oder weiterleiten.

\section{Identity and Access Management}


\todo[inline]{Federations}
\todo[inline]{OAUTH}

\section{Web Entry Server}

\todo[inline]{Reverse Proxy}
\todo[inline]{Pre-Authentication}
\todo[inline]{Filtering}
\todo[inline]{Unique ID}
\todo[inline]{Smart Filtering und URL Encryption}

\section{Server Security}

Man sollte davon ausgehen, dass die verwendete Software Schwachstellen besitzt. Auch wenn diese Schwachstellen nicht ausreichen, können sie dennoch als Sprungbrett für andere Angriffe verwendet werden. Daher ist es zwingend Nötig, das System auf allen möglichen Ebenen zu Härten und damit die Sicherheit zu erhöhen. Man spricht dabei von einer \textbf{Multi-Defense Strategy}.\\

Statistisch gesehen existiert für eine neu entdeckte Sicherheitslücke nach \textbf{6 Tagen ein Exploit} und erst \textbf{nach 54 Tagen ein Patch}. Oftmals ist Lesezugriff ausreichend, um einem Unternehmen Schaden zuzufügen oder daraus zu profitieren. Schreibzugriffe sind ebenfalls nicht zu vernachlässigen.\\

Besitzt der Angreifer Schreibzugriff, so kann er Anwendungen auf den Server laden und diese später ausführen. Beispiel: PHP Shell.

\subsection{Hardening}

\subsubsection{Während der Installation des Systems}
\begin{easylist}[itemize]
	& Minimales System
	& Keine Standard-Pfade
	& Separate Disk/Partition für log files
	& Aktuellste Patches eingespielt
	& Default Secure (Umask, Path)
	& Verwendung eines Install Servers
\end{easylist}

\subsubsection{Netzwerksicherheit}
\begin{easylist}[itemize]
	& Nur benötigte Dienste starten
	& Least File Permissions für Dienste anwenden
	& Least Process Privileges anwenden
	& Standard-/Beispielinstallationen entfernen
	& Banner Hiding (keine Versionsinformationen sichtbar)
	& Error Handling (keine Fehlerdetails sichtbar nach aussen)
\end{easylist}
\subsubsection{Authentifizierung}
\begin{easylist}[itemize]
	& Verwendung einer sicheren Authentifizierung (Benutzername, Kennwort, OTP, Client Certificate)
	& Password Policy (Stärke, Gültigkeitsdauer, Kennwortwechsel, Erkennung von Attacken auf geknackte Passwörter)
\end{easylist}
\subsubsection{Monitoring und Auditing}
\begin{easylist}[itemize]
	& Zeitsynchronisation
	& Integritätstests
	& Event handling (info, debug, error, panic, log)
	& Forensic Readiness
	& Remote Logging
\end{easylist}

\subsection{Unix Process Security}
Unter Linux ist ein weit verbreiteter Ansatz in der Prozess-Sicherheit die \textbf{Isolierung durch Chroot} (change root). Dabei wird dem Prozess ein angegebenes Verzeichnis als Root vorgegaukelt. Innerhalb des neuen Root-Verzeichnis befinden sich alle für den Service nötigen Dateien. Er kann dabei nicht auf Dateien zugreifen, welche sich ausserhalb des neuen Root-Directories besitzen. Man spricht dabei auch von einem Jail.\\

Durch Schwachstellen im Unix-Kernel ist es aber trotzdem möglich, aus diesem Jail auszubrechen.

\subsection{File Permissions}
Die Berechtigungen auf Dateisystemebene sollten so restriktiv wie möglich sein. Als Grundsatz gilt: \textbf{Never give "'world"' write access}. Dasselbe gilt für sensitive Informationen: \textbf{Keep sensitive data secure by removing read access from "'world"'}.\\

Um die Standard-Berechtigungen unter Unix zu setzen, kann mit einer Maske gearbeitet werden. Diese ist auch bekannt als "'Umask"'. Um die obigen Berechtigungen für den aktuellen Prozess zu ändern, kann z.B. \lstinline|$ umask 027|.\\

Durch dieses Vorgehen kann auch die Gefahr einer \textbf{Local File Inclusion} vermindert werden.

\begin{figure}[H]
	\centering
	\includegraphics[width=\textwidth]{./img/apache_tomcat_permissions}
	\caption{Beispiel der Berechtigungen für Apache und Tomcat}
\end{figure}

\subsubsection{Apache File Permissions}
\begin{easylist}[itemize]
	& Im Besitz von Root
	&& Konfigurationsdateien
	&& SSL Schlüssel
	&& Log
	&& Html
	& Eigentümer des Apache Prozess benötigt nur RO
	&& Html
\end{easylist}

\subsubsection{Tomcat File Permissions}
Dies gestaltet sich oftmals schwierig, denn Tomcat besitzt eine andere Prozess-Architektur. Der Eigentümer des Tomcat-Prozess benötigt RW, für Remote Deployment, Konfiguration und Load Balancing.

\subsection{Privilege Escalation}
Dabei versteht man die Möglichkeit, die Berechtigungen des aktuellen Prozesses so zu verändern, dass man Zugriff auf andere geschützte Elemente erhält. Dazu können z.B. Bugs in SetUID-Tools verwendet werden.\\

Aber auch CRON kann gefährlich sein, da die Prozesse als Root gestartet werden. Wenn nun ein von CRON-Skript "'world-writeable"' ist, kann somit schnell etwas mit höchsten Berechtigungen ausgeführt werden.

\subsection{Network Hardening}
\textbf{Nur die nötigsten Dienste} sollten gegen Anfragen von aussen reagieren. Alle anderen müssen an \textit{localhost} gebunden sein. Alternativ kann auch eine \textbf{Firewall} dafür eingesetzt werden.\\
Dienste wie UPNP, Bonjour / Zeroconf und DLNA sollten aufgrund ihrer grossen Attach-Surface auch deaktiviert werden. Dies stammt daher, dass oftmals die Geräte unzureichende Authentifizierungsmechanismen implementiert haben.\\

Es wird in den Folien auch von der \textbf{Verwendung von IPv6 abgeraten}.\\

\textbf{ICMP-Redirects deaktivieren}, um MITM-Attacken vorzubeugen.\\

Eine Analyse mittels Nmap oder \fnurl{https://cisofy.com/lynis/}{Lynis von Cisofy} helfen, Schwachstellen aufzudecken.

\subsection{DB Hardening}
Wie auf das Dateisystem trifft das Prinzip Least Privileges auch auf die Datenbank zu. Der in der Anwendung hinterlegte Benutzer darf nur auf seine Datenbanken Berechtigungen erhalten. Es können in einer Anwendung auch mehrere Benutzer hinterlegt werden, mit unterschiedlichen Berechtigungen. So z.B. ein Benutzer nur mit \textit{SELECT}-Permissions für Abfragen, \textit{INSERT} und \textit{UPDATE} für Bestellungen und ein \textit{CREATE}, \textit{DELETE} für den DB-Admin.\\
\textbf{Keinesfalls sollte der Applikations-User \textit{GRANT}-Permissions erhalten.}

\section{Mobile Application Security}

\subsection{iOS}
Basierend auf OSX hat Apple das iOS lanciert. Dabei wurde Objective-C, Swift, und C verwendet. iOS verfügt über Sandbox Mechanismen, ein Data Protection API, Code Signing und den Appstore mit manueller Freischaltung.

\begin{figure}[H]
	\centering
	\includegraphics[width=0.6\textwidth]{./img/mobileappsecurity_iOS-DataProtectionAPI}
	\caption{iOS Data Protection API}
\end{figure}

Bei iOS ist das \textbf{Permission Model} so eingerichtet, dass bei jeder Verwendung nach Erlaubnis gefragt wird und individuell erlaubt oder abgelehnt werden kann.

\subsection{Android}
Auf Linux basierend in Java und C entstand Andoid, welches von verschiedene Hardware-Anbietern übernommen wurde. Somit hat es auch noch viele alte Versionen auf dem Markt.

Android unterstützt SD Karten, Sandboxing via JVMs, Code Signing, und mehrere verschiedene Appstores.

Die \textbf{App Permissions} sind bei Android im Manifest hinterlegt und können ab Android 6 individuell angenommen oder abgelehnt werden.

\subsection{Windows Phone}
Windows Phone von Microsoft unterstützt ebenfalls strict sandboxing, Address Space Layout Randomization (ASLR, zufällige Ladeposition von Programmen in Speicher als Schutz gegen Overflow), Bit Locker Verschlüsselung, Data Protection API (ähnlich wie bei iOS), aber auch Gefahren wie automatisches Wi-Fi credential sharing über Facebook, Outlook oder Skype.

\subsection{Xamarin}
Xamarin ermöglicht die Entwicklung von platformübergreifenden, nativen Apps. Dabei wird das .NET Framework verwendet und in C\# programmiert, welches zu IL-Code (Microsoft Intermediate Language) kompiliert wird.

Die native APIs werden zwar zu den .NET namespaces gemapped, jedoch müssen platformabhängige Security Features immer noch für jede Platform geschrieben werden.

\subsection{Cordova (Phonegap)}
Cordova wird verwendet um Mobile Apps mit HTML, JS und CSS zu entwickeln. Es bringt somit native Funktionalität, basiert aber auf Web View Komponente. Somit fallen jedoch auch alle bekannten Schwachstellen an.

\section{OWASP Mobile Top 10}
\todo[inline]{OWASP Mobile Top Ten}
\begin{figure}[H]
	\centering
	\includegraphics[width=0.6\textwidth]{./img/OWASP_MobileTop10}
	\caption{OWASP Mobile Top 10, 2016}
\end{figure}

\subsection{M1: Improper Platform Usage}
Missbrauch eines Platform Features oder fehlende Sicherheitsmassnahmen, und das somitige Verletzten von Guidelines. \\

\textbf{Beispiel}: Zu viele Permissions, unverschlüsseltes Ablegen von Credentials in Property-File anstelle der KeyChain (iOS). \\

\textbf{Lösung}: Konsultieren von Guidelines beim Entwickeln.

\subsection{M2: Insecure Data Storage}
Durch Ablegen von Daten in unsicherem Speicher oder unvorhergesehenem Datenleak (siehe Unterkapitel) besteht erhöhte Gefahr wenn das Gerät gestohlen wird/verloren geht oder mit Malware infiziert ist nach jailbreak/rooting.

\subsubsection{System Screenshots}
Beim Wechseln zwischen Apps wird ein Screenshot sichtbar um zu sehen was die App zuletzt dargestellt hat. \\

\textbf{Beispiel}: E-Banking Kontoinformationen können eingesehen werden über den Screenshot des Apps. \\

\textbf{Lösung}: Screenshots vom System unterbinden oder sensitiven Daten überdecken.
\begin{lstlisting}[language=C, caption=Lösung für iOS]
(void) applicationDidEnterBackground:(UIApplication *)application{
	// mask the sensitive data or add an additional layer on top			
}
\end{lstlisting}
\begin{lstlisting}[language=Java, caption=Lösung für Android]
window.addFlags(LayoutParams.FLAG_SECURE, LayoutParams.FLAG_SECURE);
\end{lstlisting}

\subsubsection{Keyboard}
Viele Keyboards verfügen über Autocomplete. Das iOS zum Beispiel speichert bis zu 500 Wörter.\\

\textbf{Beispiel}: Passwort Feld unterstütz Autocomplete; Das Passwort wird somit auch an anderen Stellen vorgeschlagen wenn mit dem selben Buchstaben begonnen wird. \\

\textbf{Lösung}: Autocomplete bei sensitiven Daten deaktivieren.
\begin{lstlisting}[language=C, caption=Lösung für iOS]
theTextField.secureEntry = YES;
theTextField.autocorrectionType = UITextAutocorrectionTypeNo;
\end{lstlisting}
\begin{lstlisting}[language=Java, caption=Lösung für Android]
android:inputType="textNoSuggestions";
\end{lstlisting}

\subsubsection{Clipboard}
Standardmässig wird ein globales Clipboard für Copy \& Paste verwendet, und jedes App kann darauf zugreifen. \\
\textbf{Beispiel}: Sensitive Daten befinden sich noch im Clipboard aus der E-Banking App.

\textbf{Lösung}: Clipboard leeren oder ein dediziertes Clipboard verwenden.
\begin{lstlisting}[language=C, caption=Lösung für iOS]
UIPasteBoard *uniqueBoard=[UIPasteboard pasteboardWithUniqueName];
\end{lstlisting}

\subsubsection{Logs}
Crash und App logs können sensitive Daten enthalten. Im Apple System Log (ASL) bei iOS werden sogar alle Logs gecached bis zum nächsten Reboot und auch nicht sandboxed. Bei Android müssen Debug Logs bei der finalen App deaktivieren werden. \\

\textbf{Beispiel}: Bei gescheitertem Loginversuch speichert das Log Benutzername und Passwort im Log. \\

\textbf{Lösung}: Dedizierter Logger für Debug Build der App.

\subsubsection{Web Data}
Eine WebView speichert jegliche Daten wie ein konventioneller Web Browser. Dies umfasst ein Web Cache, lokaler Speicher, Cookies, Passwörter die direkt oder in eine Sqlite Datenbank gespeichert werden. \\

\textbf{Beispiel}: Ablegen von unverschlüsselten Passwörtern und Benutzernamen in Sqlite Datenbank.\\

\textbf{Lösung}: Korrekte Systemspeicher verwenden.

\subsubsection{Inter-Process Communication}
Apps können URL Handlers registrieren, der Empfänger und die überlieferten Informationen werden aber nicht überprüft. Zudem entsteht ein Problem wenn mehrere Apps für ein Handler bestehen. Bei Android kann dann die aufzurufende Applikation gewählt werden, bei iOS wird die zuletzt installierte App automatisch ausgewählt.\\

\textbf{Beispiel}: Sensitive Daten werden bei Aufruf über einen Handler abgefangen.
\begin{lstlisting}[language=XML, caption=Aufruf von Skype]
skype://090012312312?call
\end{lstlisting}

\textbf{Lösung}: Bei Android direkter Aufruf eines Intent aus der anderen App.

\subsection{M3: Insecure Communication}
Das Verwenden von inkorrekten oder alten SSL Versionen, schwache Protokollaushandlung oder einfach Plain Text Kommunikation.

Ab iOS 9 wurde mit \textbf{App Transport Security} durch mimimum Anforderungen eingeführt (TLS 1.2 mit PFS, Zertifikat und SHA-256 Signatur, 2048-bit RSA oder 256-EC Key). Apps unterstehen diesen Vorbestimmungen oder müssen explizite Ausnahmen definieren.

\subsubsection{Certificate Pinning}
Standardmässig wird bestimmten Certificate Authorities (CA) vertraut. Deren Root Zertifikate sind auf jedem Gerät vorinstalliert.\\

\textbf{Beispiel}: Die CA wurde gehacked oder ein Attackierer hat dem Gerät eine falsche CA hinzugefügt. \\

\textbf{Lösung}: Certificate Pinning, eingeführt durch Google Chrome. Speichert das Zertifikat oder dessen Public Key in der App und überprüft dieses.

\subsection{M4: Insecure Authentication}
Das Verletzten von Privacy durch Mitschicken von Gerät-spezifischen Daten, spoofing, vohersehbare Session IDs, oder kein Session Timeout.

\subsection{M5: Insufficient Cryptography}
Schwache Ciphers und/oder Keys, ungeschützte Keys in App, oder sebstgemachte Bastel-Krypto.

Android besitzt die Bouncy Castle Library. Alternativ kann dieses auch in die App eingebunden werden als Klon, welches dann Spongycastle genannt wird.

iOS hat das Common Crypto API CCCRypt. Die KeyChain kann zudem für asymmetrische Kryptographie verwendet werden. Alternativ kann OpenSSL verwendet werden.

\subsection{M6: Insecure Authorization}
Das authorisieren von unerlaubten Benutzern oder Methoden ausführen von zu wenig authorisierten Benutzern. Ein klassisches Beispiel ist Forced Browsing.

\subsection{M7: Client Code Quality}


\subsection{M8: Code Tampering}

\subsection{M9: Reverse Engineering}

\subsection{M10: Extraneous Functionality}
\subsection{Checkliste}



\section{Fraud Detection}

\todo[inline]{Panopticlick}
\todo[inline]{E-Banking}
\todo[inline]{Phishing}
\todo[inline]{Man in the middle}

\section{Security Testing}

\todo[inline]{Notwendigkeit}
\todo[inline]{Ablauf}
\todo[inline]{Begriffe}

\end{document}