\section{Einleitung}

\subsection{Information Security Management}
Kleiner Rückblick zu den Begriffen aus InfSi1:
\begin{description}
	\item[Asset] Wert aus Sicht der Organisation, ob materiell oder immateriell, Informationen oder Dienstleistungen.
	\item[Threat] Bedrohung, möglicher Grund für einen ungewollten Vorfall, der das System oder die Organisation schädigen kann.
	\item[Vulnerabilities] Schwachstelle einer Schutzmassnahme, die durch eine oder mehrere Bedrohungen ausgenutzt werden kann.
	\item[Controls] Gegenmassnahme als Mittel zur Risikohandhabung.
	\item[Gefährdung] Zusammenspiel von Asset, Threat und Vulnerabilities.
	\item[Applied Threat] Die Gefährdung ist eine Bedrohung, die konkret über eine Schwachstelle auf ein Objekt einwirkt. (Bedrohung und Schaden)
	\item[Risiko] = Wahrscheinlichkeit eines Zwischenfalls * Schaden = Bedrohung * Verletzlichkeit * Schaden
\end{description}

\subsection{Treiber für Informationssicherheit}
Als hauptsächliche Treiber der Informationssicherheit zählt die \textbf{Konformität zu Gesetz und Vorschriften}.

Folgendes liefert ein Teil von Gesetzen und Vorschriften in den verschiedenen Sparten:
\begin{easylist}[itemize]
	& Bearbeiter von Personendaten
	&& Diverse Strafgesetzbuch Artikel
	&& Datenschutzgesetz (DSG)
	&& US Health Insurance Portability and Accountability Act (HIPAA)
	& Finanzdienstleister
	&& Bankgesetze (Bankgeheimnis, etc.)
	&& EU Directive on Payment Services (PSD)
	&& Payment Card Industry Data Security Standard (PCI)
	& Telecom/ICT-Anbieter
	&& Fernmeldegesetz
	&& Lawful Interception
	& Allgemeines Controlling
	&& Sarbanes-Oxley Act (SOX), US-börsenkotierte Firmen
\end{easylist} 

\todo[inline]{Gesetzliche Anforderungen}
\todo[inline]{Bedrohungen}